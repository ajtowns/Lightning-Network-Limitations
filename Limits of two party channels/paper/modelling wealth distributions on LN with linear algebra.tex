\documentclass[10pt,twocolumn]{article}
\usepackage{cite}
\usepackage{amsmath,amssymb,amsfonts,amsthm}
\usepackage{multicol}
\setlength{\columnsep}{0.75cm}
\usepackage{caption}
\usepackage{graphicx}
\usepackage{csquotes}
\usepackage{todonotes}

\def\BibTeX{{\rm B\kern-.05em{\sc i\kern-.025em b}\kern-.08em
    T\kern-.1667em\lower.7ex\hbox{E}\kern-.125emX}}
\usepackage{hyperref}
\usepackage[margin=1.75cm]{geometry}
\usepackage[hpos=0.72\paperwidth,vpos=0.97\paperheight,angle=0,scale=0.8]{draftwatermark}
\SetWatermarkText{Preprint}
\SetWatermarkLightness{0.5}
\SetWatermarkText{Preliminary draft of work in progress}

\newtheorem{theorem}{Theorem}[section]
\newtheorem{corollary}{Corollary}[theorem]
\newtheorem{lemma}[theorem]{Lemma}


\title{Limitations of payment channels to wealth distributions of Bitcoin Users}


\author{Rene Pickhardt}
\begin{document} 
\maketitle






 
\begin{abstract}
We model the wealth distributions (aka states) of the lightning network as a discrete convex plane in a high dimensional space.
  
\end{abstract}

\section{Introduction}
The goal of this document is to twofold:
\begin{enumerate}
\item We want to see how a formulation based on linear algebra and geometry helps us to understand how payment channels limit the amount of possible wealth distributions in a network of users of bitcoins.
\item We want to generalize the formulation in order to understand how multiparty channels reduce the constraints that are put the the amount of possible wealth distributions in comparison to two party channels.
\end{enumerate}
While reading this document one should keep in mind that opening (and closing) channels are connected are secured via onchain bitcoin transactions.
Given the very limited blockspace such transactions are scarce, slow and expensive.
Generally speaking we seek a preferable\footnote{for some suitable definition of preferable} tradeoff between the expenses to maintain channels and the limits and burden they produce.
%==========================================================================
\section{The linear Space of Wealth Distributions of Bitcoin Users}
Let $U = \{u_1,\dots,u_n\}$ be the set of $n$ Bitcoin users.
We call a function $w:U\longrightarrow\mathbb{N}_0$ a wealth distribution.
In particular we write $w_i := w(u_i)$ for the wealth of a user $u_i$.
We define the capacity of the wealth function via $cap(w):= \sum_{i=1}^nw_i$.
The set
$$W_C = \{w:U\longrightarrow \mathbb{N}_0 | cap(w)=C\}$$
is called the set of wealth distributions for a fixed number of coins $C$. 
In particular we are interested in $|W_C|$ the number of wealth distributions with a fixed number of coins $C$.
From the literature we know:

$$|W_C| = {{C + n - 1}\choose {n - 1}}$$ 

If we study the $\mathbb{Z}$-algebra $\mathbb{Z}^n$ we can understand the set of wealth distributions that have a constant number of coins $C$ as the subspace defined by the linear equation:

$$w_1 + \dots + w_n = C$$

In particular each element in the subspace defined by the equation corresponds to a wealth distribution $w \in W_C$.

\section{Impact of payment channels to the Space of Wealth Distributions}
Let us assume users decide to put their wealth $w_i$ into payment channels with other users.
Each user $u_i\in U$ can allocate their wealth $w_i$ in channels with with other users $u_j$\footnote{This means that $w_{ii}=0$ $\forall i$}.
In this way the matrix $(w_{ij})_{ij}$ with the constraint
$$\sum_{j=1}^nw_{ij}\stackrel{!}{=}w_i$$
describes how the wealth is currently being distributed into payment channels between the users.

In particular for users $u_i$ and $u_j$ a payment channel $(u_i,u_j)$ of capacity $c_{ij}$ is defined by:
$$c_{ij} = w_{ij} + w_{ji}$$
From this definition we already observe the symmetry of $c_{ij} = c_{ji}$.

Under the assumption of a static network topology we have $c_{ij} = const$ $\forall i\neq j \in \{1,\dots,n\}$.
Furthermore we have:

\begin{align*}
\sum_{i=1}^n\sum_{j=1}^nc_{ij} & = \sum_{i=1}^n\sum_{j=1}^nw_{ij} + w_{ji} \\
& = \sum_{i=1}^n\sum_{j=1}^nw_{ij} + \sum_{i=1}^n\sum_{j=1}^n w_{ji} \\
& = \sum_{i=1}^n\sum_{j=1}^nw_{ij} + \sum_{j=1}^n\sum_{i=1}^n w_{ji} \\
& = \sum_{i=1}^nw_i + \sum_{j=1}^nw_j \\
& = C + C = 2\cdot C \\
\end{align*}

Because of the symmetry $c_{ij} = c_{ji}$ we have:
$$\sum_{i=1}^n\sum_{j=i+1}^nc_{ij}  = C$$

We note that a static topology of payment channels puts additional constraints to the space of wealth distributions. Thus in summary we have a subspace of $\mathbb{Z}^n$ that is defined by:

$$w_1 + \dots + w_n = C $$

as well as the additional constraints given by:
\begin{enumerate}
\item $\sum_{j=1}^nw_{ij}\stackrel{!}{=}w_i$
\item $c_{ij}=c_{ji} = w_{ij} + w_{ji}$
\item $\sum_{i=1}^n\sum_{j=i}^nc_{ij}  = C$
\end{enumerate}
\section{Questions}
\begin{enumerate}
\item as noted in a discussion with Stefan Richter the constraints $1$ and $2$ can be exchanged with the inquealities $w_i + w_j \stackrel{!}{\geq} c_{ij}$.
\item Given a set of payment channels with capacity $c_{ij}$ and $\sum_{i<j} c_{ij} = C$. \begin{enumerate}\item How large is the space of possible wealth distributions $W_C$ such that the constraints from the payment channels are respected?
\item Could we model this over $\mathbb{R}$ and just compute the volume of the subspace?\end{enumerate}
\item Assuming all coins must be put into payment channels with a static topology. How should the capacities of the payment channels for a given wealth distribution being chosen? Is the goal that the space of possible wealth distributions is maximized favourable? 
\item Assume only a limited amount of payment channels can be created (due to the fact that payment channels are achored as multisig transactions in the blockchain which has limited space). What is a desireable topology of channels at blockheight $h$ for a fixed rate of channels per block?
\item Given that a subset of the existing coins may be put in payment channels. How many coins should be used to create channels?
\end{enumerate}

\section{Multiparty channels}
While the communications protocol for multiparty channels are complicated and developer currently avoid implementing such channels several cryptographic secure constructs for multiparty channels exist.
Given the above frame work a multi party channel can most easily be described by the following constraints to the subspace of wealth distributions:

$$w_{i_1}+\dots+w_{i_k} \stackrel{!}{\geq} c_{i_1\dots i_k}$$

The number of possible $k$ party channels in a network of $n$ participants is defined by:

$${{n}\choose {k}}$$

Assuming $k$ is not fixed and allowing multi party channels of various sizses this means that
$$2^k - k -1 = \sum_{k=2}^n{{n} \choose{k}}$$
possible multi party channels can be constructed among the participants.

\subsection{Extensability argument}
It is rather simple to see that any topolgy that is created in a network of $k$-party channels is possible to realize in a network of $l$-party channels (for any $l>k$).

Assume we have $k$ party channels defined via inequalities of the form:
$$w_{i_1}+\dots+w_{i_k} \stackrel{!}{\geq} c_{i_1\dots i_k}$$
Then it is easy to extend the inequalities to:
$$w_{i_1}+\dots+w_{i_k}+w_{i_{k+1}}+\dots+w_{i_{l}} \stackrel{!}{\geq} c_{i_1\dots i_ki_{k+1}\dots i_{l}}$$
with $c_{i_1\dots i_ki_{k+1}\dots i_{l}} := c_{i_1\dots i_k}$.\footnote{I think in case indices collide the capacities of such channels just need to be summed up. However We believe this can and will only happen in edge cases and shouldn't be a problem.}

In particular we see that the in equalities for $l$ party channels allow for more degrees of freedom which means the subspace of possible wealth distributions is less constraint and thus larger.



\section{Example}
Assume we have $4$ participants bringing $3$ coins each. for $k=2$ we could create the following network:

We have $6$ payment channels of capacity $2$ each and each user is part of $3$ of the $4$ channels.
In particular a user could allocate $1$ coin to each channel.
This topology corresponds to a fully connected graph.
Each user could own at most $6$ coins in this topology.

On the other hand a setup where these $4$ participants select $k=3$ to create $3$-party channels we see that in this particular case at most $4$ such channels are possible.
Each user would be part in $3$ of the $4$ multiparty channels.
Again in this particular topology each user could have broght $1$ coin to each channel where the capacity of the channels is now $3$.
Each user could theoretically own up to $9$ coins in this instantiation of the payment channel network.

Of course fully connected networks - in particluar with a small number of users - are rather artificial. but the fact that in this particular case we need only $4$ instead of $6$ on chain transactions and users can not only have more coins but overall more wealth distributions would be possible.

\subsection{Questions}
\begin{enumerate}
\item What is a good number of $k$ or is it better not allow a flexible $k$?
\item How are the trade offs between the number of possible wealth distributions and the number of onchain transactions?
\item 
\end{enumerate}


\bibliography{linalg}
\bibliographystyle{plain}

\end {document}
